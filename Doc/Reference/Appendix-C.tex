%
% STk Reference manual (Appendix: An Introdution to STklos)
%
%           Author: Erick Gallesio [eg@unice.fr]
%    Creation date: 22-May-1994 22:13
% Last file update: 22-Apr-1998 11:04
%

\section{Introduction}

{\stklos} is the object oriented layer of {\stk}. Its implementation is
derived from version 1.3 of the Gregor Kickzales Tiny Clos
package~\cite{Tiny-Clos}.  However, it has been extended to be as close as
possible to CLOS, the Common Lisp Object System{\cite{CLtL2}}. Some features
of {\stklos} are also issued from Dylan{\cite{Dylan}} or SOS{\cite{SOS}}.

Briefly stated, the {\stklos} extension gives the user a full object
oriented system with meta-classes, multiple inheritance, generic
functions and multi-methods.  Furthermore, the whole implementation
relies on a true meta object protocol, in the spirit of the one
defined for CLOS\cite{AMOP}. This model has also been used to embody
the predefined Tk widgets in a hierarchy of {\stklos} classes. This
set of classes permits to simplify the core Tk usage by providing
homogeneous accesses to widget options and by hiding the low level
details of Tk widgets, such as naming conventions. Furthermore, as
expected, using of objects facilitates code reuse and definition
of new widgets classes.

The purpose of this appendix is to introduce briefly the {\stklos}
package and in no case will it replace the {\stklos} reference manual (which
needs to be urgently written now~\ldots).
In particular, methods relative to the meta object
protocol and access to the Tk toolkit will not be described here.

\section{Class definition and instantiation}

\subsection {Class definition}

A new class\mainindex{class} is defined with the {\tt
define-class}{\schindex{define-class} macro. The syntax of {\tt define-class}
is close to CLOS {\tt defclass}:

\begin{scheme}
(define-class \var{class} (\hyperi{superclass} \hyperii{superclass}...)
   (\hyperi{slot description} \hyperii{slot description}...)
   \hyper{metaclass option})
\end{scheme}


The \hyper{metaclass option} will not be discussed in this appendix.
The \hyper{superclass}es list specifies the super classes of \var{class} (see 
\ref{inheritance} for more details). 
A \hyper{slot description} gives the name of a slot\mainindex{slot} and,
eventually, some ``properties'' of this slot (such as its initial value, the
function which permit to access its value, \ldots). Slot descriptions will be
discussed in
\ref{slot-description}.

As an exemple, consider now that we have to define a complex number. This can
be done with the following class definition:

\begin{scheme}
(define-class  <complex> (<number>)
   (r i))
\end{scheme}

This binds the symbol {\tt <complex>} to a new class whose instances contain
two slots. These slots are called {\tt r} an {\tt i} and we suppose here
that they contain respectively the real part and the imaginary part of a
complex number. Note that this class inherits from {\tt <number>} which is a
pre-defined class ({\tt <number>} is the super class of the {\tt <real>} and
{\tt <integer>} pre-defined classes).\footnote{With this definition, a {\tt
<real>} is not a {\tt <complex>} since {\tt <real>} inherits from {\tt
<number>} rather than {\tt <complex>}. In practice, inheritance could be
modified {\em a posteriori}, if needed. However, this necessitates some
knowledge of the meta object protocol and it will not be shown in this
document}.

\section{Inheritance}
\label{inheritance}
\subsection{Class hierarchy and inheritance of slots}
Inheritance is specified upon class
definition. As said in the introduction, {\stklos} supports multiple inheritance. 
Hereafter are some classes definition: 
\begin{scheme}
(define-class A () (a))
(define-class B () (b))
(define-class C () (c))
(define-class D (A B) (d a))
(define-class E (A C) (e c))
(define-class F (D E) (f))
\end{scheme}

{\tt A}, {\tt B}, {\tt C} have a null list of super classes. In this case, the
system will replace it by the list which only contains \ide{<object>}, the
root of all the classes defined by \ide{define-class}. {\tt D}, {\tt E}, {\tt
F} use multiple inheritance: each class inherits from two previously
defined classes.  Those class definitions define a hierarchy which is
shown in Figure~1.  In this figure, the class \ide{<top>} is also shown; this
class is the super class of all Scheme objects. In particular,
\ide{<top>} is the super class of all standard Scheme types.

\begin{figure}
\centerline{\psfig{figure={hierarchy.eps}}}
\caption{A class hierarchy}
\end{figure}

The set of slots of a given class is calculated by ``unioning'' the slots of
all its super class. For instance, each instance of the class D, defined
before will have three slots ({\tt a}, {\tt b} and {\tt d}). The slots of a class 
can be obtained by the \ide{class-slots} primitive.
For instance, 
\begin{scheme}
(class-slots A) \lev (a)
(class-slots E) \lev (a e c)
(class-slots F) \lev (d a b c f)
\end{scheme}

\begin{note}
The order of slots is not significant.
\end{note}

\subsection {Instance creation and slot access}

Creation of an instance\mainindex{instance} of a previously defined
class can be done with the {\tt make}\schindex{make} procedure. This
procedure takes one mandatory parameter which is the class of the
instance which must be created and a list of optional
arguments. Optional arguments are generally used to initialize some
slots of the newly created instance. For instance, the following form

\begin{scheme}
(define c (make <complex>))
\end{scheme}

will create a new {\tt <complex>} object and will bind it to the {\tt c}
Scheme variable.

Accessing the slots of the new complex number can be done with the {\tt
slot-ref}\schindex{slot-ref} and the {\tt slot-set!}\schindex{slot-set!}
primitives. {\tt Slot-set!} primitive permits to set the value of an object
slot and {\tt slot-ref} permits to get its value.

\begin{scheme}
(slot-set! c 'r 10)
(slot-set! c 'i 3)
(slot-ref c 'r) \lev 10
(slot-ref c 'i) \lev 3
\end{scheme}

Using the \ide{describe} generic function is a simple way to see all
the slots of an object at one time: this function prints all the slots
of an object on the standard output. For instance, the expression

\begin{scheme}
(describe c)
\end{scheme}
will print the following informations on the standard output:
\begin{scheme}
\sharpsign[<complex> 122398] is an instance of class <complex>
Slots are: 
     r = 10
     i = 3
\end{scheme}

\subsection{Slot description}
\label{slot-description}

When specifying a slot, a set of options can be given to the system. Each option 
is specified with a keyword{\mainindex{keyword}}. The list of authorised keywords is 
given below:
\begin{itemize}

\item {\tt :initform}\schindex{:initform}\schindex{default slot value} 
permits to supply a default value for the slot. This default value is obtained
by evaluating the form given after the {\tt :initform} in the global
environment\mainindex{top level environment}.

\item{\tt :init-keyword}\schindex{:init-keyword} permits to specify the 
keyword for initializing a slot. The init-keyword may be provided 
during instance creation (i.e. in the make optional parameter list). Specifying 
such a keyword during instance initialization will supersede the 
default slot initialization possibly given with {\tt :initform}.

\item {\tt :getter}{\schindex{:getter}} permits to supply the name for the 
slot getter\mainindex{getter}. The name binding is done in the global 
environment\mainindex{top level environment}.

\item {\tt :setter}{\schindex{:setter}} permits to supply the name for the 
slot setter\mainindex{setter}. The name binding is done in the global 
environment\mainindex{top level environment}.

\item {\tt :accessor}\schindex{:accessor} permits to supply the name for the 
slot accessor\mainindex{accessor}. The name binding is done in the global
environment\mainindex{top level environment}. An accessor permits to get and
set the value of a slot. Setting the value of a slot is done with the extended
version of {\tt set!}\schindex{set!}.

\item {\tt :allocation}\schindex{:allocation} permits to specify how storage for
the slot is allocated. Three kinds of allocation are provided. 
They are described below:
  \begin{itemize}
     \item {\tt :instance}\schindex{:instance} indicates that each instance gets
     its own storage for the slot. This is the default.
     \item {\tt :class}\schindex{:class} indicates that there is one storage 
     location used by all the direct and indirect instances of the class. This 
     permits to define a kind of global variable which can be accessed only 
     by (in)direct instances of the class which defines this slot.
     \item {\tt :virtual}\schindex{:virtual} indicates that no storage will
     be allocated for this slot.  It is up to the user to define a getter 
     and a setter function for this slot. Those functions must be defined with the
     {\tt :slot-ref}\schindex{:slot-ref}
      and {\tt :slot-set!}\schindex{:slot-set!} options. See the example below.
  \end{itemize}
\end{itemize}

To illustrate slot description, we shall redefine the {\tt <complex>} class 
seen before. A definition could be:
\begin{scheme}
(define-class <complex> (<number>) 
   ((r :initform 0 :getter get-r :setter set-r! :init-keyword :r)
    (i :initform 0 :getter get-i :setter set-i! :init-keyword :i)))
\end{scheme}
With this definition, the {\tt r} and {\tt i} slot are set to 0 by default. 
Value of a slot can also be specified by calling {\tt make} with 
the {\tt :r} and {\tt :i} keywords. Furthermore, the generic functions {\tt get-r} 
and {\tt set-r!} (resp. {\tt get-i} and {\tt set-i!}) are automatically defined
by the system to read and write the {\tt r} (resp. {\tt i}) slot.
\begin{scheme}
(define c1 (make <complex> :r 1 :i 2))
(get-r c1) \lev 1
(set-r! c1 12)
(get-r c1) \lev 12
(define c2 (make <complex> :r 2))
(get-r c2) \lev 2
(get-i c2) \lev 0
\end{scheme}

Accessors provide an uniform access for reading and writing an object slot.
Writing a slot is done with an extended form of {\tt set!}\schindex{set!}
which is close to the  Common Lisp {\tt setf} macro. So, another definition of the
previous {\tt <complex>} class, using the {\tt :accessor} option, could be:
\begin{scheme}
(define-class <complex> (<number>) 
   ((r :initform 0 :accessor real-part :init-keyword :r)
    (i :initform 0 :accessor imag-part :init-keyword :i)))
\end{scheme}

Using this class definition, reading the real part of the {\tt c} complex can
be done with:
\begin{scheme}
(real-part c)
\end{scheme}
and setting it to the value contained in the {\tt new-value} variable 
can be done using the extended form of {\tt set!}.
\begin{scheme}
(set! (real-part c) new-value)
\end{scheme}

Suppose now that we have to manipulate complex numbers with rectangular
coordinates as well as with polar coordinates. One solution could be to
have a definition of complex numbers which uses one particular representation
and some conversion functions to pass from one representation to the other.
A better solution uses virtual slots. A complete definition of the {\tt <complex>}
class using virtual slots is given in Figure~2.
\begin{figure}
{\footnotesize
\begin{quote}
\begin{quote}
\begin{verbatim}
(define-class <complex> (<number>)
   (;; True slots use rectangular coordinates
    (r :initform 0 :accessor real-part :init-keyword :r)
    (i :initform 0 :accessor imag-part :init-keyword :i)
    ;; Virtual slots access do the conversion
    (m :accessor magnitude :init-keyword :magn  
       :allocation :virtual
       :slot-ref (lambda (o)
                   (let ((r (slot-ref o 'r)) (i (slot-ref o 'i)))
                     (sqrt (+ (* r r) (* i i)))))
       :slot-set! (lambda (o m)
                     (let ((a (slot-ref o 'a)))
                       (slot-set! o 'r (* m (cos a)))
                       (slot-set! o 'i (* m (sin a))))))
    (a :accessor angle :init-keyword :angle
       :allocation :virtual
       :slot-ref (lambda (o)
                   (atan (slot-ref o 'i) (slot-ref o 'r)))
       :slot-set! (lambda(o a)
                    (let ((m (slot-ref o 'm)))
                       (slot-set! o 'r (* m (cos a)))
                       (slot-set! o 'i (* m (sin a))))))))

\end{verbatim}
\caption{\em A {\tt <complex>} number class definition using virtual slots}
\end{quote}
\end{quote}
}
\end{figure}

\bigskip
This class definition implements two real slots ({\tt r} and {\tt i}). Values
of the {\tt m} and {\tt a} virtual slots are calculated from real slot
values. Reading a virtual slot leads to the application of the function
defined in the {\tt :slot-ref}\schindex{:slot-ref} option. Writing such a slot 
leads to the application of the function defined in the {\tt :slot-set!}\schindex{:slot-set!} option.
For instance, the following expression
\begin{scheme}
(slot-set! c 'a 3)
\end{scheme} 
permits to set the angle of the {\tt c} complex number. This expression
conducts, in fact, to the evaluation of the following expression
\begin{scheme}
((lambda o m)
    (let ((m (slot-ref o 'm)))
       (slot-set! o 'r (* m (cos a)))
       (slot-set! o 'i (* m (sin a))))
  c 3)
\end{scheme}
A more complete example is given below:
\begin{scheme}
(define c (make <complex> :r 12 :i 20))
(real-part c) \lev 12
(angle c) \lev 1.03037682652431
(slot-set! c 'i 10)
(set! (real-part c) 1)
(describe c) \lev 
          \sharpsign[<complex> 128bf8] is an instance of class <complex>
          Slots are: 
               r = 1
               i = 10
               m = 10.0498756211209
               a = 1.47112767430373
\end{scheme}

Since initialization keywords have been defined for the four slots, we
can now define the \ide{make-rectangular} and \ide{make-polar}
standard Scheme primitives.
\begin{scheme}
(define make-rectangular 
   (lambda (x y) (make <complex> :r x :i y)))

(define make-polar
   (lambda (x y) (make <complex> :magn x :angle y)))

\end{scheme}

\subsection{Class precedence list}

A class may have more than one superclass.{\footnote{
This section is an adaptation of Jeff Dalton's (J.Dalton@ed.ac.uk)
Brief introduction to CLOS)}}
With single inheritance (one superclass), it is easy to order the super classes 
from most to least specific. This is the rule:

\begin{quote}
\begin{quote}
{\bf Rule 1: Each class is more specific than its superclasses.
}
\end{quote}
\end{quote}

With multiple inheritance, ordering is harder. Suppose we have
\begin{scheme}
(define-class X ()
   ((x :initform 1)))

(define-class Y ()
   ((x :initform 2)))

(define-class Z (X Y)
   (...))
\end{scheme}

In this case, the {\tt Z} class is more specific than the {\tt X} or 
{\tt Y} class for instances of {\tt Z}. However, the {\tt :initform} 
specified in {\tt X} and {\tt Y} leads to a problem: which one
overrides the other?  The rule in {\stklos}, as in CLOS, 
is that the superclasses listed earlier are more specific than those listed later. 
So:
\begin{quote}
\begin{quote}
{\bf Rule 2: For a given class, superclasses listed earlier are more
        specific than those listed later.
}
\end{quote}
\end{quote}

These rules are used to compute a linear order for a class and all its
superclasses, from most specific to least specific.  This order is called the
``class precedence list'' of the class. Given these two rules, we can claim
that the initial form for the {\tt x} slot of previous example is 1 since the
class {\tt X} is placed before {\tt Y} in class precedence list of {\tt Z}.

This two rules are not always enough to determine a unique order, however, but
they give an idea of how things work. {\stklos} algorithm for calculating the
precedence list is a little simpler than the CLOS one described in
\cite{AMOP} for breaking ties. Consequently the calculated class precedence
list could be different. Taking the {\tt F} class shown in Figure~1, the {\stklos} 
calculated class precedence list is

\begin{quote}
{\tt
(f d e a b c <object> <top>)
}
\end{quote}
whereas it would be the following list with a CLOS-like algorithm:
\begin{quote}
{\tt
(f d e a c b <object> <top>)
}
\end{quote}

However, it is usually considered a bad idea for programmers to rely on
exactly what the order is.  If the order for some superclasses is important,
it can be expressed directly in the class definition.

The precedence list of a class can be obtained by the function 
\ide{class-precedence-list}. This function returns a ordered 
list whose first element is the most specific class. For instance, 
\begin{scheme}
(class-precedence-list B) \lev (\sharpsign[<class> 12a248] \sharpsign[<class> 1074e8] \sharpsign[<class> 107498])
\end{scheme}
However, this result is not too much readable; using the function class-name
yields a clearer result:
\begin{scheme}
(map class-name (class-precedence-list B)) \lev (b <object> <top>)
\end{scheme}

\section{Generic functions}

\subsection{Generic functions and methods}

\label{gf-n-methods}
Neither {\stklos} nor CLOS use the message mechanism for methods as
most Object Oriented language do. Instead, they use the notion of
generic function.  A generic function can be seen as a methods
``tanker''. When the evaluator requestd the application of a generic
function, all the methods of this generic function will be grabbed and
the most specific among them will be applied. We say that a method
{\em M} is {\em more specific} than a method {\em M'} if the class of
its parameters are more specific than the {\em M'} ones.  To be more
precise, when a generic funtion must be ``called'' the system will

\begin{enumerate}
\item search among all the generic function those which are applicable
\item sort the list of applicable methods in the ``most specific'' order
\item call the most specific method of this list (i.e. the first method of 
the sorted methods list).
\end{enumerate}

The definition of a generic function is done with the \ide{define-generic}
macro. Definition of a new method is done with the \ide{define-method} macro.
Note that \ide{define-method} automatically defines the generic function if it
has not been defined before. Consequently, most of the time, the
\ide{define-generic} needs not be used.

Consider the following definitions:
\begin{scheme}
(define-generic M)
(define-method M((a <integer>) b) 'integer)
(define-method M((a <real>) b) 'real)
(define-method M(a b) 'top)
\end{scheme}

The \ide{define-generic} call defines {\tt M} as a generic
function. Note that the signature of the generic function is not given
upon definition, contrarily to CLOS. This will permit methods with
different signatures for a given generic function, as we shall see
later. The three next lines define methods for the {\tt M} generic
function. Each method uses a sequence of {\em parameter specializers}
that specify when the given method is applicable. A specializer
permits to indicate the class a parameter must belong to (directly or
indirectly) to be applicable. If no speciliazer is given, the system
defaults it to \ide{<top>}. Thus, the first method definition
is equivalent to

\begin{scheme}
(define-method M((a <integer>) (b <top>)) 'integer)
\end{scheme}

Now, let us look at some possible calls to generic function {\tt M}:
\begin{scheme}
(M 2 3) \lev integer
(M 2 \schtrue) \lev integer
(M 1.2 'a) \lev real
(M {\sharpsign}3 'a) \lev real
(M {\schtrue} {\schfalse}) \lev top
(M 1 2 3) \lev error (since no method exists for 3 parameters)
\end{scheme}

The preceding methods use only one specializer per parameter list. Of course, 
each parameter can use a specializer. In this case, the parameter list is scanned 
from left to right to determine the applicability of a method. Suppose we declare
now
\begin{scheme}
(define-method M ((a <integer>) (b <number>)) 'integer-number)
(define-method M ((a <integer>) (b <real>))   'integer-real)
(define-method M (a (b <number>)) 'top-number)
\end{scheme}

In this case, 
\begin{scheme}
(M 1 2) \lev integer-integer
(M 1 1.0) \lev integer-real
(M 1 {\schtrue}) \lev integer
(M 'a 1) \lev 'top-number
\end{scheme}

\subsection{Next-method}

When a generic function is called, the list of applicable methods is built. As
mentioned before, the most specific method of this list is applied
(see~\ref{gf-n-methods}). This method may call the next method in
the list of applicable methods. This is done by using the special form
\ide{next-method}. Consider the following definitions
\begin{scheme}
(define-method Test((a <integer>))  (cons 'integer (next-method)))
(define-method Test((a <number>))   (cons 'number  (next-method)))
(define-method Test(a)              (list 'top))
\end{scheme}

With those definitions,
\begin{scheme}
(Test 1) \lev (integer number top)
(Test 1.0) \lev (number top)
(Test \schtrue) \lev (top)
\end{scheme}

\subsection{Example} 

In this section we shall continue to define operations on the {\tt <complex>}
class defined in Figure~2. Suppose that we want to use it to implement 
complex numbers completely. For instance a definition for the addition of 
two complexes could be
\begin{scheme}
(define-method new-+ ((a <complex>) (b <complex>))
  (make-rectangular (+ (real-part a) (real-part b))
                    (+ (imag-part a) (imag-part b))))
\end{scheme}

To be sure that the {\tt +} used in the method {\tt new-+} is the standard
addition we can do:
\begin{scheme}
(define-generic new-+)

(let ((+ +))
  (define-method new-+ ((a <complex>) (b <complex>))
    (make-rectangular (+ (real-part a) (real-part b))
                      (+ (imag-part a) (imag-part b)))))
\end{scheme}

The {\tt define-generic} ensures here that {\tt new-+} will be defined in the global
environment. Once this is done, we can add methods to the generic function 
{\tt new-+} which make a closure on the {\tt +} symbol.
A complete writing of the {\tt new-+} methods is shown in Figure~3.

\begin{figure}
{\footnotesize
\begin{quote}
\begin{quote}
\begin{verbatim}
(define-generic new-+)

(let ((+ +))

  (define-method new-+ ((a <real>) (b <real>)) (+ a b))

  (define-method new-+ ((a <real>) (b <complex>)) 
    (make-rectangular (+ a (real-part b)) (imag-part b)))

  (define-method new-+ ((a <complex>) (b <real>))
    (make-rectangular (+ (real-part a) b) (imag-part a)))

  (define-method new-+ ((a <complex>) (b <complex>))
    (make-rectangular (+ (real-part a) (real-part b))
                      (+ (imag-part a) (imag-part b))))

  (define-method new-+ ((a <number>))  a)
  
  (define-method new-+ () 0)

  (define-method new-+ args  (new-+ (car args) (apply new-+ (cdr args)))))

(set! + new-+)
\end{verbatim}
\caption{\em Extending {\tt +} for dealing with complex numbers}
\end{quote}
\end{quote}
}
\end{figure}

\bigskip
We use here the fact that generic function are not obliged to have the same
number of parameters, contrarily to CLOS.  The four first methods
implement the dyadic addition. The fifth method says that the addition of a single
element is this element itself. The sixth method says that using the addition with
no parameter always return 0. The last method takes an arbitrary number of
parameters\footnote{The third parameter of a define-method is a parameter list
which follow the conventions used for lambda expressions. In particular it can
use the dot notation or a symbol to denote an arbitrary number of parameters}.
This method acts as a kind of {\tt reduce}: it calls the dyadic addition on
the {\em car} of the list and on the result of applying it on its rest.  To
finish, the {\tt set!} permits to redefine the {\tt + symbol} to our extended
addition.

\bigskip
To terminate our implementation (integration?) of  complex numbers, we can 
redefine standard Scheme predicates in the following manner:
\begin{scheme}
(define-method complex? (c <complex>) \schtrue)
(define-method complex? (c)           \schfalse)

(define-method number? (n <number>) \schtrue)
(define-method number? (n)          \schfalse)
...
...
\end{scheme}

Standard primitives in which complex numbers are involved could also be redefined
in the same manner.

This ends this brief presentation of the {\stklos} extension.  
