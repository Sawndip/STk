%
% STk Reference manual (Appendix: Using the Tk toolkit)
%
%           Author: Erick Gallesio [eg@unice.fr]
%    Creation date: ??-Nov-1993 ??:??
% Last file update: 21-Jan-1996 18:29
%

When {\stk} detects that a \var{tk-command} must be called, parameters are
processed to be recognized by the corresponding toolkit function. Since the Tk
toolkit is left (mostly) unmodified, all its primitives ``think'' there is a
running Tcl interpreter behind the scene. Consequently, to work with the Tk
toolkit, a little set of rewriting rules must be known. These rules are
described hereafter.

\begin{note}
This appendix is placed here to permit an {\stk} user to make programs with
the original Tcl/Tk documentation by hand. In no case will it substitute to
the abundant Tcl/Tk manual pages nor to the excellent book by
J.~Ousterhout\cite{Ouster-book}
\end{note}

\section{Calling a Tk-command}

Since Tcl uses strings to communicate with the Tk toolkit, parameters to a
\var{Tk-command} must be translated to strings before calling the C function
which implement it. The following conversions are done, depending on the type
of the parameter that {\stk} must give to the toolkit:
\begin{quote}
\begin{description}
\item[symbol:] the print name of the symbol;
\item[number:] the external representation of the number expressed in radix 10;
\item[string:] no conversion;
\item[keyword:] the print name of the keyword where the initial semicolon has
been replaced by a dash (``-'');
\item[boolean:] the string "0" if {\schfalse} and "1" if \schtrue
\item[tk-command:] the name of the \var{tk-command}
\item[closure:] the address of the closure using the representation
  shown in \ref{addresses}.
\item[otherwise:] the external ``slashified'' version of the object.
\end{description}
\end{quote}
As an example, let us make a button with a label containing the
string {\tt "Hello, word"}. According the original Tk/Tcl documentation,
this can be done in Tcl with
\begin{scheme}
button .hello -text "Hello, world"
\end{scheme}
Following the rewriting rules expressed above, this can be done in {\stk} with
\begin{scheme}
(button '.hello '-text "Hello, world")
\end{scheme}
This call defines a new widget object which is stored in the {\stk} variable {\tt
.hello}. This object can be used as a procedure to customize our button.
For instance, setting the border of this button to 5 pixels wide and its
background to gray would be done in Tcl with
\begin{scheme}
.hello configure -border 5 -background gray
\end{scheme}
In {\stk} this would be expressed as
\begin{scheme}
(.hello 'configure '-border 5 '-background "gray")
\end{scheme}
Since keyword colon is replaced by a dash when a \ide{Tk-command} is called, this
expression could also have been written as:
\begin{scheme}
(.hello 'configure{\bf :border} 5{\bf :background} "gray")
\end{scheme}

\section{Associating Callbacks to Tk-commands}

Starting with version 3.0, {\stk} callbacks are Scheme
closures\footnote{Old syntax for callbacks (i.e. strings) is always
  supported but its use is deprecated.}. Apart scroll commands, callbacks are 
Schemes procedures without parameter. Suppose for example, that we want to
associate a command with the previous {\tt .hello} button. In
Tcl, such a command can be expressed as
\begin{scheme}
.hello configure -command \{puts stdout "Hello, world"; destroy .\}
\end{scheme}
In {\stk}, we can write
\begin{scheme}
(.hello 'configure{\bf :command} (lambda ()
                                   (display "Hello, world\backwhack{}n")
                                   (destroy *root*)))
\end{scheme}

When the user will press the mouse left button, the closure associated
to the {\tt :command} option will be evaluated in the global
environment.  Evaluation of the given closure will display the message
and call the {\tt destroy} {\em Tk-command}.

\begin{note}
\label{root window}\mainindex{root window}
The root widget is denoted ``.'' in Tcl. This convention is ambiguous with the
dotted pair convention and the dot must be quoted to avoid problems. Since
this problem arises so often, the variable \ide{*root*} has been introduced in
{\stk} to denote the Tk main window.
\end{note}

\subsection*{Managing Widget Scrollbars}

When using scrollbars, Tk library passes parameters to the widget
associated to the scrollbar (and {\em vice versa}). Let us look at a
text widget with an associated scrollbar. When the scrollbar is moved,
the command of the associated widget is invoked to change its view.
On the other side, when browsing the content of the text widget (with
arrows for example), the scrollbar is updated by calling it's
associated closure. Tk library passes position informations to
scrolling closures. This informations are the parameters of the
closure.  Hereafter is an example implementing a text widget with a
scrollbar (see the help pages for details):
\begin{scheme}
(text '.txt :yscrollcommand  (lambda l (apply .scroll 'set l)))
(scrollbar '.scroll :command (lambda l (apply .txt 'yview l)))

(pack .txt :side "left")   
(pack .scroll :fill "y" :expand \schtrue :side "left")
\end{scheme}

\section{Tk bindings}

\subsection*{Bindings are Scheme closures}

The Tk \ide{bind} command associates Scheme scripts with X events.
Starting with version 3.0 those scripts must be Scheme
closures\footnote{Old syntax for bindings (i.e. strings) is no more
  supported. Old bindings scripts must hence be rewritten.}. Binding
closures can have parameters. Those parameters are one char symbols
(with the same conventions than the Tcl \% char, see the \texttt{bind}
help page for details). For instance, the following Tcl script 
\begin{scheme}
bind .w <ButtonPress-3> \{puts "Press on widget \%W at position \%x \%y"\}
\end{scheme}
can be translated into
\begin{scheme}
(bind .w "<ButtonPress-3>" 
         (lambda (|W| x y)
            (format \schtrue "Press on widget \verb+~+A at position \verb+~+A \verb+~+A\verb+\+n" |W| x y)))
\end{scheme}

\begin{note}
  Usage of verticals bars for the \texttt{W} symbol is necessary here
  because the Tk toolkit is case sensitive ({\em e.g.}~\texttt{W}~in
  bindings is the path name of the window to which the event was
  reported, whereas \texttt{w} is the width field from the event. 
\end{note}

\subsection*{Bindings are chained}

In Tk4.0 and later, bindings are chained since it is possible for
several bindings to match a given X event.  If the bindings are
associated with different tags, then each of the bindings will be
executed, in order.  By default, a class binding will be executed
first, followed by a binding for the widget, a binding for its
toplevel, and an \texttt{all} binding.  The \ide{bindtags} command may
be used to change this order for a particular window or to associate
additional binding tags with the window (see corresponding help page
for details). If the result of closure in the bindings chain is the symbol 
\ide{break}, the next closures of the chain are not executed. The example below
illustrates this:
\begin{scheme}
(pack (entry '.e))
(bind .e "<KeyPress>" (lambda (|A|)
                        (unless (string->number |A|) 'break)))
\end{scheme}

Bindings for the entry \texttt{.e} are executed before those for its
class (i.e. \texttt{Entry}). This allows us to filter the characters
which are effectively passed to the \texttt{.e} widget. The test in this
binding closure breaks the chain of bindings if the typed character is
not a digit. Otherwise, the following binding, the one for the
\texttt{Entry} class, is executed and inserts the character typed (a
digit).
Consequently, the simple previous binding makes \texttt{.e} a
controlled entry which only accepts integer numbers.



% LocalWords:  tk Ousterhout slashified stdout

%%% Local Variables: 
%%% mode: latex
%%% TeX-master: "manual"
%%% End: 

